\chapwithshorttitle{Défis et limites}{Défis et critiques : confrontation entre théorie et pratique dans le contexte patrimonial}{Défis et critiques : confrontation entre théorie et pratique dans le contexte patrimonial}  

Tout au long de la réflexion sur l'harmonisation, la préparation et la structuration des données, plusieurs défis se sont présentés, et sont encore pour la plupart en cours de résolution. S'y intéresser va permettre d'améliorer les modèles et solutions proposées.

        \section{Une gestion des données fluctuante au rythme du projet}

La gestion du temps constitue un défi majeur lorsqu'un projet est en cours et que des données sont produites en continu. Il est capital d'anticiper et de proposer des solutions en amont pour instaurer de bonnes pratiques. Cependant, il arrive fréquemment que l'on doive intervenir en cours de route, une fois la production des données déjà entamée, comme ce fut le cas dans ce contexte. Dans ces situations, il est possible de proposer des ajustements tout en s’inspirant des éléments déjà existants, tels que des normes de nommage qui peuvent avoir été partiellement établies. Il est important de s’appuyer sur les pratiques métier des producteurs de données pour mieux comprendre leurs besoins et proposer des solutions plus harmonisées. Par ailleurs, la plupart des informations, comme les nomenclatures ou les métadonnées, peuvent encore être ajustées après coup à l’aide de logiciels spécialisés, notamment lors des phases de nettoyage des données. Garantir l’adaptation des habitudes métiers des producteurs de données va constituer le défi majeur de la gestion de ce projet. Une autre contrainte du projet réside dans le fait que la plupart des éléments proposés, du fait de l’absence de données produites, sont encore à l’état de réflexion et donc incomplets. Cela implique un ajustement progressif des propositions au fur et à mesure que les informations se précisent et que les données deviennent disponibles.\\

Par exemple, la modélisation est un travail en constante évolution, qu’elle soit de processus, conceptuelle ou relationnelle. Même si l’on peut proposer un modèle optimisé, celui-ci reste perfectible en fonction de l’évolution du projet, des technologies employées ou des informations nouvellement acquises. Par exemple, les modélisations de processus sont des documentations dynamiques qui s’affinent continuellement. Ainsi, tout ce qui concerne l’Institut de Soudure est à enrichir au fur et à mesure du projet, puisque leur rôle est de développer des techniques d’analyse. Par conséquent, les informations concernant la donnée produite par cet acteur sont encore inconnues, et seront à compléter au fur et à mesure. Les informations manquantes, souvent signalées en rouge dans les schémas, illustrent cette progression constante. Il a donc fallu accepter que la modélisation ne puisse jamais être entièrement finalisée.

    \secwithshorttitle{Difficultés techniques}{Difficultés techniques et solutions apportées}{Difficultés techniques et solutions apportées}
        \subsection{La gestion des données}
            \subsubsection{Nomenclature : entre précision ou synthèse}

La nomenclature doit permettre d’analyser un fichier de façon efficace. Les noms de fichiers ainsi proposés doivent être à la fois synthétiques et simples pour être remplis rapidement et déchiffrés aisément.\\

Une adaptation de la nomenclature des fichiers au cas par cas peut-elle être envisagée ? Certains acteurs ont émis le souhait de conserver des informations spécifiques à leurs manières de fonctionner dans leurs noms de fichier. Il sera donc nécessaire d'établir un dialogue pour s'assurer que les besoins de chacun soient pris en compte tout en convainquant de l'importance d'un nom plus concis et représentatif.\\

Actuellement, la nomenclature considère déjà différents types de documents (fichiers d’analyse, photos, imagerie 3D, documents textes). Il pourrait être judicieux de proposer une nomenclature encore plus concise, qui laisse plus de flexibilité pour renommer les fichiers. Cela serait réalisable à condition de remplir les métadonnées nécessaires. Plus le nom du fichier est court, plus il faudra remplir les métadonnées de façon méthodique et détaillée pour préserver au mieux les informations, ce qui peut dans certains cas représenter un défi, et nécessite de sensibiliser à ces pratiques.
        
            \subsubsection{Métadonnées : le choix des informations à faire figurer}

La réflexion sur les métadonnées est encore en cours : seules les techniques d'analyses les plus importantes ont été abordées. Il reste encore un certain nombre de types de données à traiter. En outre, une harmonisation avec tous les acteurs sera nécessaire pour être sûre qu’aucune information essentielle ne manquera.\\

Les métadonnées qualités n’ont pas encore été définies, seul le contenu des métadonnées descriptives a été réfléchi. Il reste également à déterminer le format et le langage dans lequel ces métadonnées seront présentées. Pour l’instant, seul leur contenu a été abordé.\\

Il pourrait être pertinent d'identifier les machines qui génèrent automatiquement des métadonnées et celles qui ne le font pas. Cela permettrait d'accorder une attention particulière aux informations à fournir lorsque les machines qui ne complètent pas automatiquement les métadonnées sont manipulées.

            \subsubsection{Tableau de gestion des données}

Le tableau de gestion des données est encore en cours de réalisation. Par conséquent, il n’y a pas encore eu de réflexion concernant la durée de conservation des documents une fois que le projet C-ADER sera fini, ni sur le tri concernant les documents à conserver ou à détruire. Cela sera à mettre en place au fur et à mesure, impliquant de s’interroger sur la valeur patrimoniale des données.\\

La plupart des colonnes rentrent dans le détail concernant certaines analyses. On trouve par exemple une colonne dédiée aux méthodes et processus de production des données, des exemples visuels des fichiers produits, des formats de traitement ou de conversion des données avec les outils et logiciels de lecture associés. Le but de ces informations est de pouvoir comprendre les données d’analyses et de les remettre en contexte. Néanmoins, on peut se demander si ce genre de détail sera laissé dans le document ou non, et s’il est envisageable de mettre ces descriptions dans d’autres documents, en annexe par exemple.

        \subsection{Accessibilité et stockage de la donnée : quels défis ?}

            \subsubsection{L’enjeu de l’accessibilité des données }

La complexité du jumeau numérique réside dans le fait qu'il a besoin d'être alimenté par un grand nombre de données pour pouvoir présenter une réflexion analytique riche. Or, l'accessibilité de certaines données peut représenter un défi dans le cadre du projet C-ADER\footcite{EssorJumeauxNumeriques2024}.

Il faut donc s'enquérir des types de données qui seront accessibles en ligne ou sur place, en tenant compte de leur caractère confidentiel et des exigences des producteurs de données. S’il s’agit de données préexistantes possédées par des tiers ne faisant pas parti du consortium (on peut penser au musée Air France par exemple qui peut détenir des archives d'avions utilisables pour la valorisation des appareils) ces dernier peuvent refuser de rendre accessible leur documents ; il sera nécessaire d’examiner les options envisageables. Il faudra alors peut-être choisir une manière d’indiquer l’existence de ces documents dans le jumeau numérique, sans pour autant les rendre accessible.\\

Dès lors, il faudra s’assurer de caviarder (masquer) les potentielles informations sensibles comme les noms, les adresses, qui seront dans les documents, conformément aux règles de confidentialité établies par les acteurs du projet C-ADER.

            \subsubsection{Plateforme de stockage et de partage des données}

Un autre défi du projet concerne le stockage des données. Quelle plateforme de partage, d’échange et de stockage des fichiers serait la plus à même de répondre aux besoins des acteurs ? Quels moyens financiers seront alloués à cette fin ? Il s’agira de faire une étude du marché afin de trouver l’outil le plus susceptible de répondre aux différents critères nécessaires pour les acteurs du projet C-ADER.\\

Par ailleurs, le plan de classement proposé sera à adapter en fonction des besoins des acteurs au fur et à mesure de l'utilisation de la plateforme de stockage. Il s’agit avant tout d’un outil de travail qui doit par conséquent rester souple et fonctionnel.\\

Globalement, il sera nécessaire d’accompagner les différent acteurs pour leur présenter les principes établis pour chaque élément, et pour les encourager à produire de la donnée propre. La documentation viendra jouer un rôle important dans cette étape, à la fois pour permettre aux producteurs de données de bien comprendre les principes à respecter, mais également pour permettre aux futurs chercheurs de bien comprendre le contexte de production de la donnée.

        \subsection{La modélisation proposée du jumeau numérique, un squelette à peaufiner}

Les deux modélisations sont chacunes incomplètes et seront développées au fur et à mesure. Ainsi, la modélisation conceptuelle ne dispose pas encore des cardinalités entre les différentes entités, c'est-à-dire le nombre de façons dont les éléments de deux groupes peuvent être liés entre eux. De même, les tables des modélisations relationnelles ne sont pas encore reliées entre elles.

Le choix d’entamer en même temps la modélisation conceptuelle et relationnelle, plutôt que de finaliser le modèle conceptuelle et de s’attaquer à l’autre est justifié par le fait que les deux modélisations se complètent. L’aspect plus concret de la modélisation relationnelle aide à préciser le contenu de la modélisation conceptuelle, tandis que les premières entités définies dans la modélisation conceptuelle servent de base à la modélisation relationnelle. Il était donc plus intéressant de procéder de la sorte.\\

Une première critique que l’on pourrait porter à cette modélisation conceptuelle est qu’elle rentre trop dans le détail et qu’elle gagnerait à être simplifiée. Certaines tables (la table « préparation de l’objet » entre autres) viennent apporter un niveau de précision peut-être trop poussé. Il faut donc garder à l’esprit que cette modélisation devra par la suite être synthétisée. En outre, de nombreux attributs, comme la longueur, la largeur ou l’épaisseur des pièces étudiées ne sont pas systématiquement indiqués, et viennent peut-être complexifier inutilement la modélisation.\\

Il sera également intéressant de tester la modélisation relationnelle proposée en créant directement la base de données, afin de voir quelles sont les propositions les plus fonctionnelles.
        
        
   