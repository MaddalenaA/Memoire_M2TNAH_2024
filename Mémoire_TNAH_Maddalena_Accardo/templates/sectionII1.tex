\chapwithshorttitle{Gestion de données : nécessité et recommandations}{La gestion des données de la recherche : entre nécessité et recommandations}{La gestion des données de la recherche : entre nécessité et recommandations}  
Une fois le contexte de production clarifié, il convient de connaître les données qui vont venir alimenter le futur outil numérique,  ainsi que le contexte dans lequel elles peuvent être utilisées.
    \secwithshorttitle{Accessibilité}{L'accessibilité : cadre légal et recommandations}{L'accessibilité : cadre légal et recommandations}

Une donnée est un élément brut (comme un mot, un nombre ou un signal) enregistré dans un système d'information, qui peut être corrélé à d'autres objets pour constituer une information. Les données peuvent être structurées, semi-structurées ou non structurées, chacune ayant des méthodes d'archivage spécifiques en fonction de leur utilisation et de leur valeur documentaire\footcite{chabinGlossaireArchivage2010}. Avant de pouvoir les exploiter, il convient de connaître le statut de ces données, et par extension, connaître le cadre juridique qui va venir influencer le traitement des données afin de mieux anticiper la manière dont elles devront êre traitées. Or, le projet C-ADER, financé par l’Agence nationale de recherche, vient s’inscrire dans un cadre à la fois public à cause de ses financements et la plupart de ses acteurs, mais aussi dans un cadre privé, puisque l’une des institutions, l’institut de Soudure, ne relève pas du public. Cette subtilité nécessite de prendre connaissance du contexte juridique.

        \subsection{Lois garantissant l’accessibilité et l’ouverture des données publiques}

Depuis plusieurs années, la France et l’Europe ont développé une politique d'ouverture et d'accessibilité des données, qui se sont concrétisés par plusieurs textes de lois.\\

En 2015, la loi Valter\footcite{LOI20151779282015} vient élargir le champ d'application de la loi CADA, en instaurant « le principe de gratuité dans la réutilisation des informations publiques ». La loi CADA\footcite{Loi7875317} garantissait déjà un début de libre accès aux documents administratifs, afin de garantir la réutilisation des informations publiques. La loi Valter vient donc renforcer les principes qui y étaient énoncés.\\

En octobre 2016, la loi pour la république numérique\footcite{LOI20161321Octobre2016}, dite Loi Lemaire, favorise à la fois « l’ouverture et la circulation des données », cherche à garantir « un environnement numérique ouvert et respectueux de la vie privée » et à faciliter « l'accès et la réutilisation des données ». Elle oblige la mise en ligne spontanée des documents administratifs librement réutilisables, et « passer d’une logique de demande citoyenne à une logique de diffusion volontaire des informations du secteur public. »\\

Il est toutefois intéressant de noter que là où les données publiques cherchent à s'ouvrir, la volonté de protéger les données personnelles se réaffirme. Le Règlement général sur la protection des données (RGPD) vient poser des limites quant aux données personnelles. Les données personnelles étaient déjà soumises à un traitement spécifique en vertu de la loi Informatique et Libertés de 1978, laquelle a été modifiée le 20 juin 2018 pour se conformer au RGPD. Le RGPD, entré en vigueur le 25 mai 2018 dans toute l'Union européenne, « instaure un nouveau cadre juridique pour la protection des données personnelles ». Il s'agit de responsabiliser les organismes et de renforcer les droits des citoyens européens. Cependant, un régime spécifique peut être mis en place pour l'utilisation des données personnelles dans le cadre des projets de recherche par les chercheurs\footcite{hadrossekGuideBonnesPratiques}.

        \subsection{Adhésion de la communauté scientifique}

Les acteurs de la recherche encouragent également l'ouverture des données par le biais de multiples recommandations. \\

En 2018, la France se dote d'un plan national pour la science ouverte, qui vient s'inscrire dans la continuité d'objectifs fixés par l'Europe. Ce plan, présenté le 4 juillet, « prône la diffusion sans entrave des publications et des données de la recherche ». Ses mesures, déclinées en trois axes, posent les conditions du développement de la science ouverte dans les établissements publics de la Recherche.\\

Dans la même lancée, le CNRS rédige en 2019 une feuille de route pour la science ouverte, structurée autour de quatre objectifs : 
\begin{itemize}
    \item 100\% de la production scientifique en accès ouvert,
    \item Développement d’une culture de la gestion et du partage des données, 
    \item Développement d’infrastructures pour la fouille et …
    \item … L’analyse des contenus et la transformation des modalités d’évaluation des chercheurs \footcite{ScienceOuverteCNRS}.
\end{itemize}

En novembre 2020, le CNRS présente un plan de gestion des données de la recherche visant à « promouvoir la science ouverte et encourager les chercheurs à rendre leurs données accessibles et réutilisables »\footcite{ScienceOuverteCNRS}. Ce plan propose, en plus de l'élaboration d'une politique des données adaptée aux besoins des communautés scientifiques, une nouvelle gouvernance ainsi qu'un plan d'action spécifique pour les données de recherche.\\

L'ANR (agence nationale de la recherche) n'est pas en reste. Dans son plan d'action 2020, celle-ci réaffirme son engagement en faveur de la science ouverte, et demande l'élaboration d'un plan de gestion des données pour les projets financés. « Partant des recommandations du Comité pour la Science Ouverte (CoSO), elle a adopté un modèle de PGD proposé par Science Europe qui vise à harmoniser la gestion des données au niveau international. Ce plan constitue désormais un livrable de tout projet financé par l’ANR [Agence nationale de recherche]. » \footcite{hadrossekGuideBonnesPratiques}

        
        \subsection{Principes FAIR : garantir la qualité et l’interopérabilité des données}

Pour assurer l'accessibilité des données et se conformer aux lois et recommandations établies, plusieurs principes permettant leur mise en œuvre efficace sont mis en avant. Les principes FAIR ont été énoncés en 2016 par des chercheurs issus du groupe de travail FORCE 11 organisé par la Fondation Lorentz à Leyde, aux Pays-Bas. Ils sont formalisés dans le cadre d'un article publié en 2016 par Mark D. Wilkinson et al., publié dans la revue Scientific Data, et intitulé \emph{The FAIR Guiding Principles for scientific data management and stewardship}\footcite{wilkinsonFAIRGuidingPrinciples2016}.

            \subsubsection{Quels sont ces principes ?}

Les projets doivent s’assurer que leurs données produites respectent quatre principes : Findable, Accessible, Interoperable, Reusable. Autrement dit : Facile à trouver, Accessible, Interopérable et Réutilisable\footcite{elisaCommentElaborerPlan2022}. Ces principes servent à garantir la science ouverte. Plus spécifiquement, ils sont conçus pour orienter les stratégies de gestion des données, aidant ainsi tous les acteurs impliqués dans leur production, la vérification de leur qualité, leur traitement et analyse, ainsi que dans leur publication et diffusion.

            \subsubsection{Facile à trouver}

Le premier principe, \textbf{facile à trouver}, suggère qu’une donnée doit être aisément trouvable aussi bien par les chercheurs que par les machines. Pour cela, il est recommandé que chaque fichier soit accompagné de métadonnées\footnote{Pour une définition précise, se rapporter au chapitre 5, III. ou voir ici : \gls{metado}}, elles-mêmes associées à un identifiant HAL, unique et pérenne, ou à un identifiant DOI. Les identifiants HAL et DOI servent à identifier de manière unique les documents scientifiques et numériques. L'identifiant HAL est spécifique à la plateforme HAL, plateforme française de dépôt et d'archivage de documents scientifiques. Il s’agit d’un code unique, attribué à chaque document déposé sur cette plateforme. Le DOI (Digital Object Identifier) est lui aussi un code alphanumérique unique utilisé pour identifier des objets numériques, tels que des articles scientifiques, des livres ou des données. C’est en revanche un système international. Il est attribué par une agence d'enregistrement et permet de retrouver le document via une URL stable : les identifiants DOI assurent une persistance des liens vers les contenus numériques, même si les URL changent.  D’autres éléments qui assurent le fait qu’une donnée soit facile à trouver sont les métadonnées. Celles-ci doivent être indexées pour pouvoir y effectuer des recherches. Enfin, l’accès à la ressource doit être libre et gratuit.
            
            \subsubsection{Accessible}

\textbf{L’accessibilité} implique qu’une donnée soit accessible et stockée de manière à ce qu’elle puisse être récupérée dans un format lisible. Pour qu’une donnée soit accessible, ses métadonnées peuvent être rendues exploitables via des protocoles ouverts et standards ; on entend par là des ensembles de règles et de spécifications conçus pour garantir l'interopérabilité et la compatibilité entre différents systèmes et technologies. Les plus connus dans le cadre de la recherche sont OAI-PMH, API, RDF Triplestore, OAIS, ou via des API de type REST ouvertes. Ils assurent la gestion, l'interaction et l'échange de données entre systèmes et applications. Ainsi, OAI-PMH est centré sur la collecte de métadonnées, les API (Application Programming Interface) - un ensemble de règles permettant à des logiciels distincts de communiquer et d'interagir entre eux - permettent une interaction entre logiciels, RDF Triplestore est une base de données qui structure ses données en triplets ce qui permet d’y effectuer des requêtes, OAIS est un modèle de référence pour l'archivage numérique, et les API REST offrent une méthode flexible pour l'accès aux services web par le biais de méthodes HTTP. Les données doivent être stockées dans un environnement sécurisé, et les documents archivés afin de préserver leur accessibilité et leur lisibilité sur le long terme.

            \subsubsection{Interopérable}

\textbf{L’interopérabilité} désigne la capacité à combiner et utiliser des données avec d’autres jeux de données. Plus précisément, elle correspond à la faculté d'échanger des informations et à assurer le fonctionnement entre différents outils numériques. Pour qu’une donnée soit interopérable, il faut privilégier les langages et les formats ouverts. L’utilisation d’identifiants, à l’image d’identifiants HAL ou DO évoqués précédemment est recommandé : ils permettent la citation et le suivi des ressources à travers diverses plateformes. De même des alignements avec des référentiels peuvent être bénéfiques puisque ces derniers associent des entités à des identifiants reconnus et standardisés. Enfin, l’utilisation de vocabulaires standards, tels que Dublin Core (DC), Resource Description Framework (RDF), ou encore Friend of a Friend (FOAF) pour ne citer que quelques exemples, permettrait de structurer les métadonnées de façon uniforme.
            
            \subsubsection{Réutilisable}

Enfin, une donnée est réutilisable lorsqu’elle est préparée dans l’objectif d’une utilisation future par d’autres personnes ou pour de nouveaux usages. Une donnée réutilisable dispose de métadonnées qui fournissent des informations sur la provenance des données ainsi que leur réutilisation. La provenance, le contexte et les conditions d’utilisation sont, dans l’idéal, indiqués. C’est dans ce cadre qu’on peut par exemple réfléchir à un type de licence pour préciser les droits liés à la diffusion des données\footcite{PrincipesFAIR}.\\ 

Désormais, les universités et les établissements de recherche ne peuvent plus invoquer leur droit en tant que producteurs de bases de données pour empêcher la libre réutilisation des informations qu'ils génèrent. Le principe d’ouverture s’applique par défaut. Ce principe ne concerne pas seulement les documents, mais les jeux de données et les bases de données elles-mêmes. Néanmoins, il est important de ne pas confondre ces principes avec des obligations strictes. Dans certains cas, la divulgation des données peut être limitée en raison de la présence d'informations personnelles ou de leur lien avec des brevets industriels. Cela peut être justement le cas des données de l’institut de Soudure dans notre contexte. Il faut alors trouver un juste milieu. C’est dans ce contexte que vient s’appliquer le principe « aussi ouvert que possible, aussi fermé que nécessaire ». Le choix des licences viendra alors par exemple nuancer l’ouverture des données, de même que les restrictions ou les délais d’ouvertures (embargos) à la fin d’un projet.\\

Justement, le projet C-ADER se heurte à des difficultés auprès de certains membres du consortium pour convaincre de la nécessité d’ouvrir, à terme, les données. Ces obstacles soulignent l'importance de sensibiliser davantage les acteurs concernés aux bénéfices de cette démarche.

    \secwithshorttitle{Plan de gestion des données}{Assurer un suivi des données produites : le plan de gestion des données (PGD)}{Assurer un suivi des données produites : le plan de gestion des données (PGD)}

Pour respecter ces principes, la rédaction d’un plan de gestion des données peut être approprié. Il vient soulever un certain nombre de questions susceptibles d’être posées dans le cadre d’un projet afin d’assurer le meilleur traitement de la donnée possible. C'est donc, dans le cadre du projet C-ADER, un élément clé.

        \subsection{Objectifs et utilité} 

Un plan de gestion de données (PGD) ou data plan management (DMP) et un outil essentiel pour garantir l’intégrité scientifique d’un projet.\\

Il s'intéresse à toutes les données générées : tous les éléments observés, produits, ou collectés par le chercheur. Cela peut comprendre des textes et des corpus de textes, des images, des photographies, des modèles numériques en 3D, des données d’entraînement en intelligence artificielle, des enregistrements audiovisuels, des données d’observation, des bases de données, et plus encore. La notion de données englobe aussi l’ensemble des codes sources, des méthodes et des protocoles utilisés pour présenter, analyser, instrumenter ou diffuser ces éléments, qui soutiennent la recherche et peuvent servir de preuves pour les interprétations des chercheurs. Dans le cadre du PGD, le terme de donnée a donc été étendu jusqu’à celui de jeu de données. Un jeu de données (dataset en anglais) se définit comme une collection de fichiers électroniques qui présentent une certaine cohérence et qui sont assemblés pour former un ensemble unifié. L'échelle à laquelle cette agrégation est réalisée ainsi que les critères utilisés sont laissés à la discrétion des scientifiques. Ces critères peuvent varier significativement en fonction des questions de recherche, de la nature des données, des équipements employés, ou des possibilités de réutilisation\footcite{elisaCommentElaborerPlan2022}.\\

Un PGD détaille la manière dont les données seront traitées tout au long de leur cycle de vie. Son but est de disposer d’une vue d'ensemble sur les données, en expliquant les modalités de collecte, de contenu et d'organisation des données, ainsi que leur accès, partage et réutilisation future. Ces connaissances rendent possible l’anticipation de potentiels problèmes techniques ou juridiques. En créant une vision commune au sein d'une équipe, le PGD facilite l'intégration de nouveaux collaborateurs et peut servir de document de référence en dressant ainsi le portrait des données. Fort de ces informations, il peut également servir d'outil de pilotage et d’aide à la décision\footcite{elisaCommentElaborerPlan2022}.\\

Enfin, le PGD est un document évolutif. L'Agence Nationale de la Recherche (ANR) n'exige pas une version définitive d’un PGD dès le début d’un projet, mais attend qu'il en accompagne le déroulement, s'ajustant aux besoins et aux avancées de la recherche au fur et à mesure de ses différentes étapes. Il est donc constamment mis à jour pour enregistrer les informations concrètes liées à la mise en œuvre des décisions et pour s’adapter à l’évolution de la recherche. À la fin du projet, enrichi de toutes les informations recueillies, il devient un élément essentiel pour comprendre le contexte de production des données et constitue un historique du projet documenté.

        \subsection{Contenu et enjeux}

Les points soulevés ici sont des enjeux soulevés par le plan de gestion. Ils n’appellent pas tous à être répondus, mais visent plutôt à ouvrir une réflexion sur le traitement des données.

            \subsubsection{Connaître la donnée}
Il s’agit tout d’abord de prendre connaissance de l’intégralité des données amenée à être produites dans le cadre d’un projet.\\

Il est essentiel de connaître le type, le format et le volume des données - une estimation peut suffire - qui seront collectées, étudiées, générées ou utilisées au cours d’un projet de réalisation d’un jumeau numérique. Cela inclut également l’identification des données existantes qui seront réutilisées. Les standards, méthodes et mécanismes d’assurance qualité appliqués jouent un rôle clé pour garantir un contrôle optimal et une bonne connaissance des données. De plus, il est important de planifier l’organisation des fichiers et la gestion de leurs différentes versions, en définissant comment les données seront administrées tout au long du projet. Cela comprend les conventions de nomenclature, le suivi des versions et l’arborescence des dossiers.\\

Dans le cadre du projet C-ADER, cela revient par exemple à détailler les conditions de production des résultats d’analyses et la chaîne de production de la donnée en fonction des différentes machines utilisées.\\

Il faut également déterminer la documentation et les métadonnées qui seront fournies. En effet, tous les types de données qui seront fournies doivent dans l’idéal être décrites afin d’aider les futurs utilisateurs à comprendre et réutiliser les données. La documentation doit permettre aux utilisateurs, qu'ils soient humains ou machines, de lire et interpréter les données ultérieurement. Il convient également d'expliquer comment cette documentation sera générée, et comment les données seront produites à l’aide de métadonnées. Enfin, il est important d'indiquer les standards définis par la communauté qui seront adoptés pour annoter les données et métadonnées, s'ils existent.\\

Il est également central de définir la documentation et les métadonnées qui seront fournies. Idéalement, toutes les données doivent être décrites de manière à aider les futurs utilisateurs à les comprendre et à les réutiliser. La documentation doit venir aider les utilisateurs, qu'ils soient humains ou machines, à lire et interpréter les données ultérieurement. Il est recommandé d’expliquer comment cette documentation sera générée et comment les données seront enrichies par les métadonnées. Enfin, le plan de gestion de données permet de préciser les standards communautaires qui seront adoptés pour annoter les données et les métadonnées, si de tels standards existent.\\

C’est la partie du plan de gestion de données qui sera la plus traitée au cours du stage.

            \subsubsection{Stocker et conserver}

Un deuxième enjeu du plan de gestion de données se concentre sur le stockage et la préservation des données.\\

En effet, il est nécessaire de connaître les capacités de stockage disponibles ainsi que l'emplacement physique où les données seront conservées. Cela peut passer par l’indication de mise en œuvre de procédures de sauvegarde, comprenant la fréquence des mises à jour, les responsabilités des différents intervenants, des procédures automatiques ou manuelles, ainsi que des mesures de sécurité mises en place pour protéger les données.\\

Il est également nécessaire de définir un plan précis pour la conservation des données après la conclusion du projet, afin d’éviter tout problème lié à la perte des données ou à leur stockage. Cela comprend l'identification des procédures utilisées pour sélectionner les données à conserver, partager, archiver ou supprimer, mais également la spécification des critères qui guideront cette sélection, tels que la valeur à long terme des données, leur potentiel de réutilisation, ou les obligations de destruction éventuelles. Un plan de gestion invite également à déterminer quels formats de fichiers seront employés pour la conservation à long terme des données, et de justifier ces choix en fonction des standards communautaires et des exigences spécifiques du projet.

            \subsubsection{Le contenu des données et leur sécurité}

Un troisième enjeu porte sur les questions éthiques, judiciaires et de sécurité des données dans le cadre d’un projet.\\

Les questions éthiques peuvent entraîner une adaptation des pratiques quant à la gestion des données. Dans ce cadre, il peut être judicieux de déterminer quels standards de protection s'appliquent aux données utilisées et si des clauses de confidentialité sont en vigueur. Il convient également de vérifier si toutes les autorisations nécessaires ont été obtenues pour collecter, traiter, conserver et partager les données. Il est important de s'assurer que les personnes dont les données sont réutilisées ont été informées ou ont donné leur consentement.\\

En outre, il est essentiel de prendre en compte la sécurité et l’accès aux données. Cela comprend l’identification des préoccupations principales en matière de sécurité, d’évaluer les risques associés à chaque aspect de la gestion des données, et de mettre en place des mesures de protection spécifiques pour atténuer ces risques. La régulation des droits d’accès aux données est un autre moyen pour y parvenir. Il peut également être nécessaire de définir des permissions d’accès spécifiques. Pour les données personnelles ou sensibles, il est impératif de détailler les mesures de sécurité appliquées pour le stockage et le transfert. Enfin, dresser une liste des normes officielles adoptées et décrire les procédures ou dispositifs utilisés pour assurer le stockage et le traitement sécurisés des données sensibles est recommandé.\\

Une réflexion sur les droits d'auteur et la propriété intellectuelle doit être également menée. Il faut déterminer qui sera le propriétaire des données collectées et générées durant le projet de recherche, et spécifier les licences qui régiront l’utilisation et la réutilisation des données. Il est aussi important de définir les restrictions concernant la réutilisation des données appartenant à des tiers pour respecter les droits de propriété intellectuelle existants. Au sein du projet C-ADER, ces éléments ont par exemple été déjà abordés, et ont abouti à un consortium identifiant les propriétaires des données, et par extension les conditions de diffusion de ces mêmes données.

            \subsubsection{Partage et réutilisation}

Un quatrième enjeu concerne le partage et la réutilisation des données, autrement dit, comment les données seront partagées et accessibles.\\

Pour cela, il faut déterminer le dépôt dans lequel les données seront stockées et mises à disposition, et clarifier comment les utilisateurs pourront accéder aux données qu’ils ont produites. Il est également important de préciser comment d'autres chercheurs pourront signaler et valoriser la réutilisation de ces données.\\

De plus, un facteur clé supplémentaire est la protection des données sensibles et les conditions de leur mise à disposition. Il est essentiel de déterminer s'il existe des restrictions nécessaires pour protéger les données sensibles, en tenant compte de clauses légales, éthiques, de copyright ou de confidentialité qui pourraient s'appliquer. Les données doivent être partagées dès que possible, idéalement au moment de la publication des résultats scientifiques. Tout retard éventuel dans la mise à disposition des données doit être motivé par des raisons légitimes, et il convient d'évaluer si un accord de confidentialité pourrait assurer une protection adéquate des données confidentielles.\\

Un autre élément important est la protection des données sensibles et les conditions de leur accès. Il faut déterminer les restrictions nécessaires pour protéger ces données en respectant les clauses légales, éthiques, de copyright ou de confidentialité applicables. . Les données doivent être partagées dès que possible, idéalement au moment de la publication des résultats scientifiques. Tout retard éventuel dans la mise à disposition des données doit être motivé par des raisons légitimes. De même, ces questions ont été abordées dans le consortium du projet C-ADER, qui a ainsi fixé une période d’embargo après la fin du projet en 2027, afin d’assurer un délai suffisant pour la publication de ses résultats scientifiques.\\

La majorité des éléments d’un plan de gestion s’alignent avec les principes FAIR ou, à tout le moins, soulèvent des questions visant à garantir leur respect.