\chapwithshorttitle{Cas d'usage : industrie, patrimoine}{Cas d'usage des jumeaux numériques : de l’industrie au patrimoine}{Cas d'usage des jumeaux numériques : de l’industrie au patrimoine}  

Si le concept de jumeau numérique s'est rapidement développé ces dernières années, il est intéressant de consulter quelques exemples, aussi bien dans le cabre industriel que patrimonial, pour mieux comprendre à quoi l'outil numérique du projet C-ADER sera susceptible de ressembler à la fin du projet.

        \secwithshorttitle{Dans l'industrie}{Développement et usage dans le secteur industriel}{Développement et usage dans le secteur industriel}

C’est dans le secteur industriel qu’est très largement portée l’utilisation du jumeau numérique. Il est intéressant de se pencher sur les utilisations qui en sont faites, dans le secteur aéronautique principalement, mais également dans d'autres secteurs afin de saisir l'importance de cette technologie de nos jours, et expliquer son engouement pour le projet C-ADER.\\

Les entreprises les plus prolifiques dans ce domaine sont General Electric, avec Predix ; Dassault Systèmes, avec sa plateforme 3Dexpérience, et Siemens, avec le logiciel SIMIT, dont le nom revient de nombreuses fois dans les exemples de jumeaux proposés. Leurs noms reviendront ainsi souvent dans les exemples proposés.\\

Un des premiers secteurs à utiliser le jumeau numérique comme outil de travail est, dans la foulée de l’aérospatial, l’automobile et l’aéronautique. Le choix a été fait de ne se pencher que sur l'aéronautique, puisque c secteur croise également le coeur du projet C-ADER.\\

À l’occasion du salon du Bourget en 2015, le patron de PTC, James Heppelmann faisait la démonstration d’un jumeau numérique à partir d’une simple maquette posée sur une table, devant les journalistes. Celle-ci représentait l’avion en exploitation, équipé de capteurs d’ouverture des portes, de thermomètres, d’un gyroscope, etc. Au mur, un écran affichait son modèle numérique répercutant en temps réel les données provenant de la maquette. L’objectif était d’avoir une représentation numérique fidèle de l’appareil exploité par le client\footcite{gladieuxJumeauNumeriqueAeronautique2019}. La même année, Airbus développait sa première maquette numérique complète d'un avion, l'A350, qui était mis en service. L'A321 connaît lui-aussi la création d'un jumeau numérique pour le pré-assemblage de son fuselage. L'entreprise continue d'utiliser cette technologie pour assurer une continuité digitale entre le design et l'utilisation des produits, en se basant sur la plateforme 3DExperience de Dassault Systèmes – plébiscitée par Airbus comme par tous les grands acteurs du secteur, tels que Boeing, Safran, Lockheed Martin–. Les bénéfices sont multiples pour les avionneurs.\\ 

Toujours d'actualité, le concept de jumeau numérique est le cœur du contrat de maintenance Ravel, pour Rafale verticalisé, signé en mai 2019 entre Dassault Aviation et la Direction de la maintenance aéronautique du ministère des Armées, pour une durée de dix ans. Dassault Aviation doit mettre en œuvre le jumeau numérique pour l’ensemble des équipements – hors moteurs et sièges éjectables qui font l’objet de contrats distincts – des 152 Rafale de l’armée de l’air et de la Marine nationale afin d’améliorer leur maintien en condition opérationnelle (MCO). Les données de maintenance des flottes sont récupérées puis introduites dans le jumeau numérique de chaque appareil via la plate-forme 3DExperience de Dassault Systèmes, avant d'être analysées.\footcite{olivierAvionneursEmparentJumeau2020}\footcite{MirorringRealityDigital2024}\footcite{MiroirMiroirJumeau2021}.\\

Le but de ces jumeaux numérique est avant tout d'en faciliter la construction. Ils visent à anticiper et optimiser le choix des matériaux ou des pièces réalisées, grâce à la réalisation de simulations pour mieux les tester. Ils permettent également d'assurer la maintenance des appareils représentés\footcite{attaranDigitalTwinsIndustrial2024}. Le grand succès des jumeaux numérique dans l'aviation justifie ainsi le souhait du projet C-ADER de voir, à terme, un outil similaire se réaliser.\\

Le jumeau numérique fleurit également dans d'autres domaines. La logistique s'y intéresse, dans la mesure où il permet de représenter une usine entière et de la piloter, comme c'est le cas pour l'usine de l’entreprise Solvay située à Champalé\footcite{parisotCreerJumeauNumerique2015}\footcite{lawtonPremiersJoursJumeaux2021}. Le secteur du bâtiment, qui utilise déjà un outil numérique similaire, le BIM, acronyme de building information modeling, y trouve également un grand intérêt dans la gestion des villes ; on pense notamment à l'exemple type de Virtual Singapore généré en 2015 \footcite{walkerSingaporeDigitalTwin2023}, ou encore à l'échelle française, à la ville de Rennes\footcite{RennesMetropole2017} en 2017, qui ont développé chacune leur propre jumeau numérique. Enfin, le secteur de la santé s'intéresse lui aussi depuis peu aux possibilités offertes par le jumeau numérique. Une vingtaine d’approches médicales sont développées depuis 2012 sur la plateforme 3DExperience\footcite{jaliniereJumeauxNumeriquesNouveaux2021}et de nouveaux projets fleurissent, tels que le projet EDITH en 2022\footcite{SanteEcosystemeEuropeen2024}, dans le but d'étudier les possibilités de l'outil.\\

Ces différents cas viennent donc mettre en lumière la diversité des approches, des objectifs et des outils utilisés. Cette diversité est tout autant d'actualité lorsqu'il s'agit du secteur patrimonial.

        \secwithshorttitle{Dans le patrimoine}{Jumeaux numériques dans le domaine patrimonial}{Jumeaux numériques dans le domaine patrimonial}

Principalement utilisés dans la représentation de bâtiments emblématiques ou d’œuvres d’art mythiques, les jumeaux numériques développés par des institutions patrimoniales ont pour objectifs non plus la fabrication, la gestion ou la maintenance d’un produit, mais davantage la mise en valeur et la conservation de l’objet représenté, ainsi que l’accès aux données correspondantes. Contrairement à d'autres applications axées sur l'anticipation des futurs états des systèmes, l'utilisation des jumeaux numériques dans un contexte patrimonial se concentre sur la reconstitution ou la préservation du passé. 

          \subsection{Etat des lieux et perspectives }

L’usage des jumeaux numériques dans le domaine patrimonial est plus récent ; on trouve très peu de cas de jumeaux numériques recensés. En revanche, l’usage de technologies associées, à savoir la modélisation ou  la représentation 3D ainsi que le principe de réalité virtuelle tend à se démocratiser\footcite{debideranVisitesNumeriquesParcours2014} et à ouvrir la voie à l'utilisation future de jumeaux numériques. En effet, la plupart des institutions patrimoniales et culturelles ont pris conscience dès le début des années 2010 du potentiel offert par ces technologies.\\

Un pionnier dans l'utilisation de ces technologies est le site Google Art \& Culture\footcite{ExplorerGoogleArts}. Le géant d'internet a conclu plusieurs partenariats avec de grandes institutions qui lui donnent accès à leurs collections qu'il met en valeur. Quelques œuvres font l'objet d'un traitement particulier -qui peut se rapprocher de la conception du jumeau numérique- : celui d'une numérisation dont le résultat aboutit à la représentation en 3D de l'objet, et enrichi par des informations. C'est le cas notamment du David de Michel Ange\footcite{MichelangeloDavid3D}, qui avait déjà fait l'objet d'un traitement similaire dans le cadre d'une exposition en 2020 à Dubaï visant à reproduire une copie grandeur réelle en résine de la statue après plusieurs scans\footcite{lydonMichelangeloDavidHas2021}\footcite{farooqiMichelangeloDavidHas2021}. S'il ne s'agit pas encore d'un jumeau numérique, l'intérêt pour cette technologie va être rapidement exprimé.\\ 
 
C'est le cas notamment du British Museum. Grâce à son partenariat avec Google Art \& Culture, le musée avait déjà rendu accessibles et librement consultables un certain nombre de ses collections\footcite{BritishMuseumLondon}. Ses œuvres sont mises en valeur en étant accompagnées d'explications visant à les remettre en contexte, suivant le principe de visite virtuelle assez répandu dans les institutions muséales telles que le musée londonien « Victoria and Albert Museum »\footcite{ExploreCollections}. Le British Museum a cependant exploité toutes les possibilités offertes par Google Art \& Culture, en reprenant le principe de « street view »\footcite{BritishMuseumLondon}. Il est ainsi possible de se promener dans ses 85 galeries donnant accès aux collections permanentes. Le British Museum souhaite cependant pousser ses expérimentations numériques plus loin et a exprimé son intérêt pour les avantages que pouvaient offrir un jumeau numérique pour ses collections. Le projet est en cours, et un premier pas a été fait côté modélisation : le musée a scanné et mis à disposition plus de 200 objets d'arts scannés en 3D\footcite{BritishMuseum}, et consultables gratuitement, telle la célèbre pierre de rosette\footcite{RosettaStoneDownload2017}.\\ 

Ces possibilités, et tout particulièrement la modélisation 3D et l'agrégation de données, permettent donc de retranscrire la matérialité d’une œuvre à un degré dépassant celui d’une photographie en 2D. Une représentation 3D fait figurer les volumes, mais aussi -lorsque la qualité de la numérisation le permet-, les textures d’un objet, créant l’illusion de manipuler l’œuvre et de l’observer de très près, là où la muséographie ne le permet pas forcément pour des contraintes de conservation évidentes.\\

Outre les objets de collection, ce sont les sites et bâtiments patrimoniaux qui font eux aussi l'objet d'attentions particulières. En effet, de nombreux projets visent à mettre en valeur et rendre accessibles un patrimoine inconnu ou difficile d'accès.\\

CyArk\footcite{CyArk} cherche ainsi à rendre le patrimoine mondial accessible par le biais des technologies numériques 3D. Plusieurs sites sont ainsi reproduits numériquement en se basant sur le principe de photogrammétrie terrestre et aérienne, enrichis de documentations permettant de les consulter et d'en apprendre davantage, le but étant de favoriser une expérience immersive, mais aussi de permettre aux professionnels du patrimoine et de la conservation de s’en servir. Le projet invite enfin au libre échange des données, suivant le principe de l'open data. Bien que représentant une prouesse technologique et offrant d'excellentes résolutions de sites archéologiques ou patrimoniaux, comme en témoigne l’exemple de Pétra\footcite{AdDeirMonastery}, le site est davantage conçu pour le grand public que pour les professionnels de la conservation. En effet, la documentation des éléments privilégie une perspective adaptée aux besoins des amateurs plutôt qu'à ceux des scientifiques et des conservateurs.\\

C'est encore le cas avec les recherches menées par le projet ScanPyramids, qui vise à étudier la structure de quatre pyramides (les pyramides Sud, pyramides Nord, pyramides de Kheops et de Sephren) à l'aide de thermographie infrarouges et de radiographies. A long terme, le but est d'en proposer une modélisation 3D disponible en open data\footcite{ScanPyramids}.\\

En outre, les visualisations 3D permettraient sur le long terme de préserver les objets ou bâtiments patrimoniaux dont l'exposition répétée à la présence humaine accélérerait la dégradation. De nombreux projets ont ainsi pour ambition de scanner et de représenter numériquement de vastes espaces fragilisés, pour alléger la présence humaine tout en permettant de les découvrir. C'est le cas par exemple de la grotte de Lascaux. Un jumeau virtuel a été réalisé en partenariat avec Dassault Systèmes et la Cité de l'architecture \& du patrimoine pour permettre aux visiteurs de déambuler dans les 235 mètres de galeries de la grotte\footcite{VisiteGrotteLascaux}, représentées en grandeur nature.\\ 

Dans une optique de recherche et de conservation, le site de Pompéi a également été soumis à une campagne intensive de scans dans le cadre du projet Pompeii I.14 Project. Les recherches archéologiques et les représentations numériques sont combinées et réunies sur un seul outil représenté en 3D -un jumeau numérique- afin de fournir un maximum de détails exploitables et consultables par les chercheurs. L'objectif est de centraliser l'information et de continuer à visualiser les ruines et les données associées, tout en reconstituant la ville\footcite{Pompeii14Project}\footcite{ModelingPompeiiNew2023}.\\

Enfin, l’usage d’une technologie numérique 3D associé à l’agrégation de données permettrait de reproduire et préserver les strates temporelles d’un élément pour visualiser son évolution dans le temps. Le but est de disposer de toutes les informations concernant un même objet ou bâtiment.\\

Ce concept est repris pour les études concernant Notre Dame de Paris. Il s'agit ici de disposer, en un seul outil, du plus grand nombre d'informations sur la cathédrale et de ses strates temporelles. Cette initiative proposée après la catastrophe de 2019,  vise à permettre aux architectes du patrimoine de disposer d'une représentation fidèle du bâtiment d'entamer la reconstruction, mais surtout d'agréger les données concernant la cathédrale, format ainsi une gigantesque base de donnée concernant la cathédrale « Le groupe de travail [\dots] prépare un « jumeau numérique » qui regroupe tout ce que nous savons sur la structure—des esquisses de construction aux scans 3D de son état actuel—et qui sera également capable d'intégrer toutes les données et informations futures. »\footcite{veyrierasDigitalTwinNotreDame2019}.

	       \subsection{Conception du jumeau numérique par les acteurs du projet C-ADER }

Le jumeau numérique du Boeing 707 et des autres avions sélectionnées vient s’inscrire dans la continuité de ces projets patrimoniaux. À cheval sur plusieurs sujets, tels que l'aéronautique et l'histoire, l'usage d'un jumeau numérique pour le projet C-ADER est d'autant plus justifié que ces deux domaines y ont de plus en plus recours.\\

Une représentation 3D de chaque appareil sélectionné dans le cadre du projet serait réalisée. Le but principal et de pouvoir localiser où les données d'analyse ont été faites sur la partie précise de l'avion, et par extension, de la modélisation qui le représenterait. Chaque partie de l’avion, qu’il s’agisse des ailes, du fuselage ou de la dérive serait interactive (comprendre : cliquable), et afficherait les données et les documents relatifs à cette partie d'avion précise. Toutes les données techniques, historiques, scientifiques et autres seraient ainsi non seulement réunies en un même point, mais recontextualisées car replacées précisément sur la partie de l'avion concernée.\\ 

L'enjeu pour les différents acteurs du projet est de pouvoir situer les informations obtenues tout au long des analyses sur les zones d’analyses d’un avion. Dans une optique future, il s'agirait de rendre accessible cette représentation enrichie d'un avion ainsi que des données correspondantes autant que faire se peut.\\

Le jumeau numérique souhaité dans le cadre du projet C-ADER représenterait soit un avion complet, soit les parties de l’avion scannées et sur lesquelles auraient été réalisées des analyses ; le choix reste ouvert et sera précisé au fur et à mesure de l’avancement du projet et des besoins précis des acteurs.\\ 
             
De même, le degré de précision dans la modélisation est encore lui aussi à définir. La modélisation 3D viendrait avant tout aider à positionner les analyses sur l’avion, et à montrer les informations qui y sont relatives, avant de faire office de représentation visuelle à l’image du David de Michel-Ange par exemple.\\

Bien que constituant le socle de tout le volet concernant la valorisation et la diffusion des données, il est encore difficile de dresser une description précise du jumeau tel qu'il sera proposé à la fin du projet. Le projet est encore à ses débuts, la production des données d’analyses, de même que la collecte des données historiques ne commencent à être effectuées que dans le courant de l’année 2024. Avant d’aboutir à une proposition plus précise, il s’agit de prendre connaissance des données qui vont venir alimenter l’outil, sous quelle forme, et de les préparer.\\

Une fois la technologie comprise, et les projets de valorisation énoncé, il convient d'élaborer une gestion de la donnée qui puisse garantir son exploitation future. Il s’agit pour cela de proposer des outils pour répondre à des besoins de partage et de stockage de données entre acteurs.\\