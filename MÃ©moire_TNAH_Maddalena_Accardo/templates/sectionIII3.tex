\chapwithshorttitle{Redéfinition du jumeau numérique}{Retour d’expérience : vers une redéfinition du jumeau numérique}{Retour d’expérience : vers une redéfinition du jumeau numérique}  

est-ce réellement un jumeau numérique qui est souhaité dans le cadre du projet C-ADER ? Enfin, malgré les efforts de clarifications proposés, la définition du jumeau numérique reste encore floue et sujette à confusion. Il s'agit d'un outil idéal, offrant de multiples perspectives, mais qui semble davantage être une vision rêvée qu'une réalité pleinement réalisable, et qui est encore en cours de développement. Dans le domaine du patrimoine, on constate ainsi le développement d'une terminologie spécifique qui reflète mieux l'orientation du projet C-ADER.

    \secwithshorttitle{Un concept délicat et versatile}{Un concept délicat et versatile : aux côtés du jumeau numérique, de nombreuses technologies similaires souvent confondues}{Un concept délicat et versatile : aux côtés du jumeau numérique, de nombreuses technologies similaires souvent confondues}

Tout un questionnement sur l'usage de la terminologie du jumeau numérique, et sur ce qui était réellement souhaité techniquement (au point même de ne plus correspondre à la définition d'un jumeau numérique) s'est posé dans le cadre du projet C-ADER. 

En effet, l’expression « jumeau numérique » reste assez floue. Ce terme tend de nos jours à être utilisé de manière extensive, souvent employé pour valoriser divers outils de modélisation et capter l'attention des auditeurs ou des clients\footcite{PremiersJoursJumeaux}. Du fait des multiples définitions et interprétations qui en ont été faites, reconnaître un jumeau numérique peut aisément prêter à confusion. De nombreux outils numériques sont ainsi confondus avec les jumeaux numériques, bien qu'ils n'offrent pas les mêmes fonctionnalités et qu'ils ne correspondent pas à la définition technique qui en a été faite.

        \subsection{Le jumeau numérique, comme dépassement de la simulation et de la modélisation ?}

Il est très courant de confondre un jumeau numérique avec d'autres notions proches telles que la simulation numérique, ou encore la modélisation 3D, qui l'ont précédés. Ces éléments, similaires sur certains points à un jumeau numérique, ont alimenté la confusion autour de ce terme.\\

La simulation numérique est une méthode informatique qui permet de prévoir comment un système ou un processus fonctionnera à un moment donné. En simulant comment un système réagit aux différentes situations, elle aide à prédire son comportement dans des conditions particulières. Une simulation se restreint à n’étudier qu’un processus particulier, alors qu’un jumeau numérique peut s’appliquer à simuler un bien plus grand nombre d’éléments. De plus, le degré de précision du jumeau numérique est bien plus élevé que celui d’une simulation.\\

Un jumeau numérique se distingue également des maquettes 3D. Une modélisation 3D cherche à reproduire un modèle virtuel qui capture la forme, la taille et les détails de l'objet avec une grande précision. Cela permet de visualiser et d'explorer l'objet sous tous les angles. Or, un jumeau numérique ne se juge pas au degré de qualité de la visualisation représentée. Au contraire, ce degré de visualisation peut être plus ou moins fin en fonction de ce qui est recherché, puisque ce sont les données qui l’alimente et qu’il produit qui compte, et non l’aspect visuel. Aucun jumeau numérique ne peut être considéré comme une reproduction totale et parfaite de son équivalent physique, que ce soit au niveau de la modélisation, de la simulation ou des données injectées.\\

Or, un jumeau numérique vient s’inscrire dans la continuité de ces outils numérique. Bien qu’il prolonge ces technologies, il les intègre et les dépasse en offrant des fonctionnalités plus avancées\footcite{bealJumeauNumeriqueRealite}. C'est la raison pour laquel il est souvent confondu avec ces différents outils.

        \subsection{Trois dénominations pour un outil}

Le jumeau numérique est souvent confondu avec des éléments quasi similaires, comme en témoignent les nombreuses expressions utilisées pour le désigner : double numérique, jumeau virtuel, etc. Deux termes se détachent plus particulièrement : modèle numérique et ombre numérique. Krintzinger affirme même :  

\begin{quote}
    « Based on the given definitions of a Digital Twin in any context, one might identify a common understanding of Digital Twins, as digital counterparts of physical objects. Within these definitions, the terms Digital Model, Digital Shadow and Digital Twin are often used synonymously\footnote{Traduction : « En se basant sur les définitions données du jumeau numérique dans divers contextes, on peut identifier une compréhension commune des jumeaux numériques comme des répliques numériques d'objets physiques. Dans ces définitions, les termes « modèle numérique », « ombre numérique » et « jumeau numérique » sont souvent utilisés de manière interchangeable. »}. »\footcite{kritzingerDigitalTwinManufacturing2018}
\end{quote}

 Un jumeau numérique, au vu de ses multiples définitions en fonction des contextes ou des industries qui les produisent, est généralement défini comme la version digitale d’un objet physique. Cette flexibilité dans la compréhension du terme explique ainsi que les termes modèle numérique, ombre numérique et jumeau numérique sont souvent utilisés de façon interchangeable. C’est par exemple le cas dans le cadre du projet C-ADER. En effet, le jumeau numérique est à considérer comme la perspective finale à développer au terme du projet. Or, l’analyse des données effectuée ainsi que les réflexions sur sa modélisation et les données qui vont l'alimenter ont permis d’établir un état des lieux afin d’en anticiper la réalisation. C'est dans ce contexte que les concepts de modèle numérique ou d’ombre numérique ont été souvent mentionnés.\\

Un \textbf{modèle numérique} est une représentation visuelle numérique d'un objet ou d'un système. Le transfert des données, que ce soit du modèle physique vers le modèle réel, et vice-versa, nécessite l’intervention humaine et n'est pas automatisé. Il n'a pas de connexion dynamique avec le monde physique et ne permet pas l'échange automatique de données en temps réel. Il ne se met pas à jour tout seul et ne communique pas directement avec l'objet réel. Ce modèle n'exclut cependant pas la possibilité de réaliser des simulations ou des analyses, mais se base uniquement sur les données qui lui ont été fournis sans avoir la possibilité de les mettre à jour et de les renouveler automatiquement. \\ 

Une \textbf{ombre numérique} est également une représentation numérique d'un objet ou d'un système. Toutefois, une ombre numérique reçoit les informations du modèle physique qu'elle représente automatiquement, ce qui lui donne la capacité de se mettre à jour et de le représenter en temps réel. Cependant, contrairement au jumeau numérique, le contraire n'est pas possible : un changement du modèle virtuel n'est pas répercuté sur le modèle physique. De par son appellation, il semblerait que le terme « ombre numérique » soit souvent perçu comme moins prometteurs dans le patrimoine et la culture que son équivalent, le « jumeau numérique », bien que ces deux concepts ne désignent pas les mêmes outils et répondent à des besoins distincts. \\

En comparaison, un \textbf{jumeau numérique} est une réplique virtuelle d’un objet ou d’un système physique qui assure un échange automatisé et bidirectionnel de données entre le monde réel et son modèle numérique. Ce concept se distingue par une interaction continue où les informations circulent à la fois du monde physique vers le modèle virtuel et inversement, créant ainsi une boucle dynamique.  Contrairement à des représentations statiques ou partiellement connectées comme le modèle numérique ou l’ombre numérique, le jumeau numérique non seulement reflète l’état actuel de l’objet réel, mais permet également d’influencer ce dernier en temps réel. Les modifications apportées au jumeau numérique peuvent avoir un impact direct sur l’objet physique, et inversement, toute évolution de l’objet réel est répercutée automatiquement dans le modèle numérique. Le jumeau numérique représente ainsi le niveau d’intégration le plus abouti\footcite{donniniToutComprendreJumeau2023}\footcite{kritzingerDigitalTwinManufacturing2018}.

\begin{table}[ht!]
\centering
\small 
\begin{tabularx}{\textwidth}{|>{\centering\arraybackslash}X|>{\centering\arraybackslash}X|>{\centering\arraybackslash}X|>{\centering\arraybackslash}X|}
\hline
\textbf{Propriétés} & \textbf{Jumeau numérique} & \textbf{Ombre numérique} & \textbf{Modèle numérique} \\
\hline
\emph{Représentation visuelle} & \textbf{Oui} & \textbf{Oui} & \textbf{Oui} \\
\hline
\emph{Reflet de l’état de l’élément physique en temps réel} & \textbf{Oui} & \textbf{Oui} & Non \\
\hline
\emph{Mise à jour automatique} & \textbf{Oui} & Non & Non \\
\hline
\emph{Si changement du modèle physique, alors changement du modèle virtuel} & \textbf{Oui} & \textbf{Oui} & Non \\
\hline
\emph{Si changement du modèle virtuel, alors changement du modèle physique} & \textbf{Oui} & Non & Non \\
\hline
\end{tabularx}
\caption{Comparaison des trois technologies}
\label{tab:Comparaison des trois technologies}
\end{table}

Le choix de la technologie la plus à même de répondre aux besoins du projet sera à déterminer lorsque le projet tirera à sa fin. D’ici là, le travail d’analyse et de collecte des données se doit de rester suffisamment large pour s'adapter à tout modèle choisi, et d'assurer avant tout les échanges de données entre partenaires. Seule la modélisation sera à surveiller et amenée à évoluer en fonction de la technologie choisie.\\

     \secwithshorttitle{Patrimoine : nouvelles potentialités}{Jumeau numérique et patrimoine, des potentialités renouvellées et en continuel développement ?}{Jumeau numérique et patrimoine, des potentialités renouvellées et en continuel développement ?}
 
        \subsection{Perspective : intelligence artificielle et jumeau numérique, atout pour le patrimoine ?}

Pour assurer la réalisation de données prédictives, un jumeau numérique s’appuie majoritairement sur de l’intelligence artificielle. C'est ce qui permet de dépasser la simple représentation et aboutit à la prédiction et l'analyse de données. Le choix de l'intelligence artificielle (IA) dans les jumeaux numériques dépend du domaine d'application spécifique. Trois principales approches d'IA sont souvent utilisées :\\

\begin{itemize}
\item \textbf{Machine Learning (Apprentissage Automatique)} : C'est une méthode où un ordinateur apprend à partir de données pour faire des prédictions ou des décisions. Il ne faut pas le programmer pour chaque tâche spécifique ; au lieu de cela, il trouve lui-même des modèles ou des tendances dans les données pour accomplir des tâches.
\item \textbf{Deep Learning (Apprentissage Profond)} : C'est une forme avancée de machine learning qui utilise des réseaux de neurones, inspirés du cerveau humain. Ces réseaux sont capables de traiter des informations complexes, comme des images ou du texte, en passant par plusieurs étapes pour reconnaître des motifs - par exemple identifier des objets dans une photo.
\item  \textbf{Reinforcement Learning (Apprentissage par Renforcement)} : Dans cette approche, un agent (comme un robot ou un logiciel) apprend en faisant des essais et en recevant des récompenses ou des pénalités selon ses actions. Au fil du temps, il s'améliore en ajustant ses actions pour maximiser les récompenses et atteindre un objectif optimal\footcite{jinBigDataMachine2024}.\\
\end{itemize}

L’utilisation de cet outil dans le secteur patrimonial serait très précieuse. A l’échelle des institutions patrimoniales par exemple, on pourrait envisager d’utiliser l’intelligence artificielle pour reconstituer des objets d’arts partiellement détruits. Sans pour autant chercher à restaurer l’objet (ce qui n’est pas dans les logiques patrimoniales actuelles), cela pourrait avoir un intérêt, lorsqu’il s’agit notamment de reconstituer des bâtiments ou des appareils, de suivre leur évolution et leur possible dégradation sur le long terme, ou d’anticiper les processus de dégradation. Un usage semblable de l’intelligence artificielle commence ainsi à être mis en oeuvre, non pas dans le cadre d’un jumeau numérique toutefois, à la BNF, où un algorithme cherche à anticiper l’état sanitaire des collections de la bibliothèque.\footcite{leroyterquemIntelligenceArtificielleAu}.\\

Les possibilités qu’offrirait l’intelligence artificielle si elle était implémentée à l’outil numérique représentant les avions du projet C-ADER sont également intéressantes. Les données d’analyses de corrosion et les données relevant la température du Musée de l’Air et de l’Espace où sont entreposés les avions choisis pour le projet permettraient d’aboutir à la création d’un algorithme dont le but serait de prédire la rapidité moyenne de la corrosion d’un appareil par exemple.\\ 

Cependant, l’utilisation de l'intelligence artificielle n'est pas indispensable pour le déroulement immédiat du projet C-ADER. Même si à terme la réalisation d’un jumeau numérique est souhaitée, sa conception telle qu'elle est comprise à l'heure actuelle ne correspond pas tout à fait à la définition qui est faite d’un jumeau numérique, à savoir un modèle 3D évoluant en même temps que son homologue physique, alimenté par des données fournies en temps réel venant de l'avion exposé, et capable de produire des données en fonction de celles qu'il a ingéré. De même, il n’est pas prévu à ce jour que des données viennent alimenter en temps réel un double numérique d’un des avions du Musée de l’Air et de l’Espace dans le cadre du projet, mais que ces données soient ajoutées ponctuellement. Enfin, il n’est pas question d’utiliser ces données pour obtenir des analyses prédictives ou prescriptives -bien que ces perspectives soient très intéressantes-.\\ 

En réalité, la perspective de jumeau numérique qui devra être générée au terme du projet a bien plus une vocation patrimoniale que prédictive. Par conséquent, il serait plus juste d’avancer l’idée que le choix d’un modèle ou d’une ombre numérique sera à terme plus pertinent. Néanmoins, quel que soit l’outil numérique élaboré à terme, le choix dépendra des modalités de réalisation de l'outil numérique, du choix final des institutions souhaitant valoriser leur projet, ainsi que des possibilités offertes en fonction des contraintes budgétaires et temporelles. La question du jumeau numérique des avions du projet C-ADER s’inscrit donc davantage comme un objectif vers lequel les acteurs tendent, plutôt qu’une dénomination figée désignant rigoureusement une technologie précise.

        \subsection{entre jumeau numérique et objet patrimonial augmenté}

Néanmoins, l'intelligence artificielle, qui confère au jumeau numérique toute sa puissance, trouve également des limites. L'intelligence artificielle n'est pas immunisée contre les biais et dépend de la qualité de ses données pour fournir des résultats satisfaisants. En outre, certains questionnent la capacité -et la finalité- du jumeau numérique à agréger trop de données. Un « mille-feuilles » d’informations ne signifie pas forcément que le rendu prédictif et l'analyse obtenue soit plus efficace\footcite{EssorJumeauxNumeriques2024}. Or, dans le domaine patrimonial, la volonté est d’agréger le plus d’informations possibles afin de les rendre accessibles et de les mettre en valeur, non pas tant de mettre à profit cette capacité prédictive.\\  

De plus, les éléments désignés comme « jumeau numérique » dans le secteur patrimonial ne correspondent donc pas toujours pleinement à la définition d'un jumeau numérique. En effet, les diverses initiatives patrimoniales présentées auparavant ne cherchent pas à maintenir une représentation continue d’un lieu, d’un processus ou d'un objet, mais plutôt à créer un modèle capturant un instant T unique, enrichi de multiples informations. Qu’il s’agisse du projet ScanPyramids ou des réalisations de CyArk, les mises à jour des éléments modélisés ne se font pas en temps réel, mais sont effectuées manuellement par les gestionnaires et les acteurs des projets.\\

 Ainsi, il semble que l’expression de jumeau numérique, très en vogue dans les projets culturels actuels, soit souvent utilisée sans que le produit fini ne corresponde réellement à cette définition. Le concept de jumeau numérique demeure en pratique un idéal vers lequel tendent de plus en plus les institutions, se manifestant de manière croissante dans la valorisation des collections, des objets et des sites patrimoniaux. Il s'agit presque d'une interprétation différente du concept par les institutions patrimoniales qui l'utilisent, une interprétation qui se rapproche de la définition du jumeau numérique telle qu'elle était proposée au début des années 2000, sans données prédictives, et qui est mieux adaptée à leurs besoins spécifiques. \\

Plus apprécié que le jumeau numérique dont la technologie se prête plus facilement aux éléments industriels que patrimoniaux, c’est l’objet patrimonial augmenté qui gagne en popularité dans le secteur culturel.
Un objet patrimonial augmenté ou patrimoine augmenté fait référence à l'utilisation des technologies numériques pour enrichir, étendre ou compléter la manière dont un objet patrimonial (comme une œuvre d'art, un monument, un site historique, etc.) est perçu, étudié, et expérimenté. 
L’obtention d’un tel objet patrimonial implique la fusion des données qui lui sont liées, leur organisation ainsi que leur visualisation.
Là où un jumeau numérique est une réplique virtuelle utilisée pour surveiller et optimiser un objet ou un système physique, un objet patrimonial augmenté enrichit l'expérience des artefacts culturels avec des éléments numériques pour offrir des interactions et des informations supplémentaires.\\ 

L’objectif de ce concept vise à dépasser la seule belle animation 3D pour enrichir, voire renouveler la connaissance ; il est pour cela semblable au jumeau numérique. Il repose également sur la séparation de la base de connaissances et de l’interface de consultation de l’objet. Ce parti pris permet d’assurer d’une part l’évolution des contenus du dispositif selon l’apport de nouvelles recherches et/ou préoccupations (et en ce sens ce type de projet est en perpétuelle progression) ; d’autre part, ce type de structuration des données permet d’envisager le développement d’autres interfaces (mobiles, réalité augmentée, etc.) à partir de la même base de connaissances\footcite{gasnierTechnHomTime2020}.\\

Le projet Espadon, acronyme pour « En Sciences du Patrimoine, l’Analyse Dynamique des Objets anciens et Numériques » se focalise sur la création de tels objets\footcite{EquipExESPADONFondation}. Le projet vise à relever des défis dans l'analyse des objets patrimoniaux en améliorant leur caractérisation grâce à des approches 2D et 3D, ainsi qu’en fusionnant des données hétérogènes pour en extraire de nouvelles informations. Il cherche également à renforcer les capacités de traitement, de gestion, de stockage et d'échange de données massives, en cohérence avec la transition numérique en cours. ESPADON ambitionne de créer un écosystème numérique interopérable pour représenter et accéder aux informations patrimoniales, en développant le concept d'« objet du patrimoine augmenté », qui servira de nouveau média pour la science du patrimoine. \\

C’est le cas également du projet Techn’hom Time qui souhaite reconstituer numériquement à l’aide d’une représentation spatio-temporelle 3D un ancien quartier industriel, aujourd’hui connu sous le nom de Techn’hom, à Belfort\footcite{gasnierTechnHomTime2020}.\\

L’utilisation d’un jumeau numérique ou de la mise en place d’éléments pour favoriser et mettre en valeur des objets patrimoniaux reste néanmoins à l’appréciation au cas par cas en fonction des objets en question et des besoins exprimés. Le jumeau numérique pourrait, dans une certaine mesure, n’être plus considéré que comme un objet patrimonial augmenté. \\